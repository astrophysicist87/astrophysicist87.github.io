\documentclass[12pt, tightenlines, onecolumn, showpacs, amsfonts, aps, prc, nofootinbib, floatfix]{revtex4-2}
\pagestyle{empty}


\begin{document}


\large
\noindent Christopher Plumberg%
\hfill%
\textit{Statement of Teaching Philosophy}%
\vspace{-12.5pt}
\rule{\columnwidth}{1pt}

\medskip

\normalsize


I was in graduate school when I first encountered the concept of a ``toy model."

\medskip

At first, I remember the term sounding strange to me.  The way my professors typically used the phrase, a ``toy model" was just a model of some physical mechanism which was unusually simple or clear in its basic elements.  So why not just call it a ``simple model," or something like this?  Why call it a \textit{toy} model?

\medskip

The conclusion I eventually came to was this: a ``toy model" is so called because it connotes the disposition to \textit{play}.  In other words, a ``toy model" is not just a model of some mechanism with solely pragmatic value; it is a \textit{toy} model: something to be explored, tweaked, broken, fixed, tested, and viewed from every possible angle, in order to learn as much as one can from it.  It is a minimal, descriptive model of some physical mechanism which is simple enough to contemplate and learn from and enjoy for its own sake.

\medskip

Eventually, I came to realize that the disposition to play formed an essential aspect of my own research and learning process, a realization which has heavily influenced the way that I teach.  At its core, a healthy approach to physics is about a love of learning and discovery, and this is exactly the attitude that I try to cultivate in my classroom.  Although a course in physics entails a certain minimal body of knowledge or know-how which the successful student should acquire during it, at least as important as \textit{what} the student learns is \textit{how} the student develops in the acquisition of knowledge as a life-long learner.  A healthy approach to physics is thus with an attitude of ownership and discovery - that is, with an attitude of play.

\medskip

Emphasizing an attitude of play is central to an inquiry-based learning approach and therefore affects several aspects of my teaching style.  These aspects can be formulated succinctly as \textit{what}, \textit{how}, and \textit{who}.  Let me discuss each of these in turn. 

\medskip

\noindent \underline{1. \textbf{What:} \textit{Course and lecture content}}

\medskip
	
	In addition to presenting the basic physical content which is relevant to a given course, I also try to communicate `meta-strategies' which are relevant to problem solving in \textit{any} field.  For instance, in addition to helping introductory physics students grasp the basics of two-dimensional kinematics, I also try to embody an attitude of play for them by encouraging them to understand the problem from a variety of angles before attempting to solve it.  Strategies for doing this include (but are not limited to): drawing a picture to help visualize the problem; labeling the available pieces of information and identifying which of them are relevant to the problem at hand; clarifying what the problem is asking for and formulating strategies for obtaining the solution; and so on.  In this way, I encourage my students not merely to memorize rules for solving physics problems, but to develop general strategies for solving new problems in the future.

\medskip

These problem-solving strategies are naturally reinforced by including a laboratory component in the course.  This offers students essential ``hands on" experience with the novel concepts they encounter in the classroom.  I can still remember, in my own physics education, how the intuition behind ``pulley problems" never entirely made sense until I actually constructed and solved a real-life pulley problem in the laboratory.  Something about seeing the physical concepts illustrated concretely enables a sense of active engagement with the subject and enhances the sense of ownership and discovery which the student needs to succeed.

%\medskip
\newpage

\noindent \underline{2. \textbf{How:} \textit{Lecture style and student progress evaluation}}

\medskip
	
	Encouraging and embodying an attitude of play affects not only \textit{what} material I present in my courses, but \textit{how} I present this material as well.
	
\medskip

	My lectures always begin with an illustrative example, interspersed with Socratic-style leading questions, in order to build intuition for what we should expect the solution to be like.  For instance, when teaching Gauss' law in electrostatics, I start by helping the students to visualize electric flux (e.g., by counting field lines) and showing them pictorially how it relates to the net electric charge.  The point is to help the students to develop an internal sense of the concepts that the mathematical description and solution of the problem will address.  Beginning with an illustration in this way - by relating the problem itself to concepts and experiences that are already familiar to the students - involves the students actively in the problem's solution, and thereby invites an attitude of discovery and excitement.

\medskip

	The attitude of play also influences how I evaluate my students' progress in the course.  I have found it especially useful to assign a weekly problem for which, in addition to providing the worked solution to the problem itself, the students must also supply a creative, independent check of their solution.  In one class, for example, when solving a problem requiring the velocity-versus-time graph of a rocket's launch and subsequent free-fall back to earth, several students checked their solutions by integrating their graphs by hand and showing explicitly that the net displacement of the rocket was zero.  The element of creativity required here demonstrates the need for an attitude of play: the students who excel will be the students who have learned how to direct their enjoyment and creativity toward devising a clear, precise, and accurate solution which passes reasonable checks.  
	
\medskip

\noindent \underline{3. \textbf{Who:} \textit{Maintaining a healthy and inclusive classroom environment}}

\medskip

Finally, cultivating an attitude of play requires fostering a sense of belonging and an inclusive environment where students from different backgrounds can engage with the relevant concepts in the way which comes most naturally to them.  One way I have found to do this is by featuring prominently in my lectures contributions of physicists from a wide variety of backgrounds, together with brief biographical sketches, in order to enable all students to see themselves as well represented in the field of physics.  Presenting the work of scientists like Emmy Noether, Marie Curie, and Carolyn Parker in this way, alongside that of Niels Bohr, Enrico Fermi, and Albert Einstein, emphasizes the historical contributions of underrepresented groups to physics and encourages the attitude of play by reinforcing each student's sense of belonging and ability to contribute to the field.

\medskip

One particularly effective approach I have found to building and maintaining a welcoming environment is to address all of my students by their names, and if possible, to commit them to memory.  The small effort to learn a student's name and call them by it encourages active participation in the classroom environment.  Students have occasionally even told me directly how much they appreciated this effort on my part.  Creating an environment where students feel known and recognized gives them the freedom to learn in the way which is most natural for them, and thereby provides a place in which discovery and play can happen naturally.

\medskip

The attitude of play has thus shaped many aspects of my teaching style and philosophy.  I strive to maintain a classroom environment where students feel comfortable exploring and discovering physics for themselves, without sacrificing the attainment of rigorous and excellent work.  Each student, in turn, not only acquires competency in physics, but also develops the best practices and habits of a life-long learner.


\end{document}
